%%%%%%%%%%%%%%%%%%%%%%%%%%%%%%%%%%%%%%%%%%%%%%%%%%%%%%%%%%%%%%%%%%%%%%%%%%%%%%%
%
%   This file is part of FLINT.
%
%   FLINT is free software; you can redistribute it and/or modify
%   it under the terms of the GNU General Public License as published by
%   the Free Software Foundation; either version 2 of the License, or
%   (at your option) any later version.
%
%   FLINT is distributed in the hope that it will be useful,
%   but WITHOUT ANY WARRANTY; without even the implied warranty of
%   MERCHANTABILITY or FITNESS FOR A PARTICULAR PURPOSE.  See the
%   GNU General Public License for more details.
%
%   You should have received a copy of the GNU General Public License
%   along with FLINT; if not, write to the Free Software
%   Foundation, Inc., 51 Franklin St, Fifth Floor, Boston, MA  02110-1301 USA
%
%%%%%%%%%%%%%%%%%%%%%%%%%%%%%%%%%%%%%%%%%%%%%%%%%%%%%%%%%%%%%%%%%%%%%%%%%%%%%%%
%%%%%%%%%%%%%%%%%%%%%%%%%%%%%%%%%%%%%%%%%%%%%%%%%%%%%%%%%%%%%%%%%%%%%%%%%%%%%%%
%
%   Copyright (C) 2007 William Hart, David Harvey
%   Copyright (C) 2010 Sebastian Pancratz
%
%%%%%%%%%%%%%%%%%%%%%%%%%%%%%%%%%%%%%%%%%%%%%%%%%%%%%%%%%%%%%%%%%%%%%%%%%%%%%%%

\documentclass[a4paper,10pt]{book}

\title{FLINT 2.0: Fast Library for Number Theory}
\author{AUTHORS}

%%%%%%%%%%%%
% geometry %
%%%%%%%%%%%%

\usepackage[hmargin=3.8cm,vmargin=3cm,a4paper,centering,twoside]{geometry}
\setlength{\headheight}{14pt}

% Dutch style of paragraph formatting, i.e. no indents
\setlength{\parskip}{1.3ex plus 0.2ex minus 0.2ex}
\setlength{\parindent}{0pt}

%%%%%%%%%%%%%%%%%%
% Other packages %
%%%%%%%%%%%%%%%%%%

\usepackage{amsmath,amsthm,amscd,amsfonts,amssymb}
\usepackage{cases}
\usepackage[all]{xy}

\usepackage{ifpdf}
\usepackage{paralist}
\usepackage{fancyhdr}
\usepackage{sectsty}
\usepackage{natbib}
\usepackage{url}
\usepackage[T1]{fontenc}
\usepackage{ae,aecompl}
\usepackage{booktabs}
\usepackage{multirow}
\usepackage{verbatim}
\usepackage{listings}

%%%%%%%%%%%%
% hyperref %
%%%%%%%%%%%%

\usepackage{hyperref}
\hypersetup{
    colorlinks=true,    % false: boxed links; true: colored links
    citecolor=green,    % color of links to bibliography
    filecolor=magenta,  % color of file links
    linkcolor=red,      % color of internal links
    urlcolor=blue       % color of external links
}

\makeatletter
\newcommand\org@hypertarget{}
\let\org@hypertarget\hypertarget
\renewcommand\hypertarget[2]{%
    \Hy@raisedlink{\org@hypertarget{#1}{}}#2%
} 
\makeatother

\ifpdf
    \hypersetup{
        pdftitle={FLINT},
        pdfauthor={},
        pdfsubject={Computational mathematics},
        bookmarks=true,
        bookmarksnumbered=true,
        unicode=true,
        pdfstartview={FitH},
        pdfpagemode={UseOutlines}
    }
\fi

%%%%%%%%%%
% natbib %
%%%%%%%%%%

\bibpunct{[}{]}{,}{n}{}{}

\renewcommand{\bibname}{References}

%%%%%%%%%%
% secsty %
%%%%%%%%%%

\allsectionsfont{\nohang\centering}

%%%%%%%%%%%%
% fancyhdr %
%%%%%%%%%%%%

\newcommand\nouppercase[1]{{%
    \let\uppercase\relax
    \let\MakeUppercase\relax
    \expandafter\let\csname MakeUppercase \endcsname\relax#1}%
}

\pagestyle{fancyplain}

\renewcommand{\chaptermark}[1]{\markboth{#1}{}}
\renewcommand{\sectionmark}[1]{\markright{\thesection\ #1}}
\fancyhf{}
\fancyhead[LE,RO]{\bfseries\thepage}
\fancyhead[LO]{\itshape\nouppercase{\rightmark}}
\fancyhead[RE]{\itshape\nouppercase{\leftmark}}
\renewcommand{\headrulewidth}{0pt}
\renewcommand{\footrulewidth}{0pt}

\fancypagestyle{plain}{%
  \fancyhead{}
  \renewcommand{\headrulewidth}{0pt}
}

\makeatletter
\def\cleardoublepage{\clearpage\if@twoside \ifodd\c@page\else
    \hbox{}
    \thispagestyle{plain}
    \newpage
    \if@twocolumn\hbox{}\newpage\fi\fi\fi}
\makeatother \clearpage{\pagestyle{plain}\cleardoublepage}

%%%%%%%
% url %
%%%%%%%

\makeatletter
\def\url@leostyle{%
  \@ifundefined{selectfont}{\def\UrlFont{\sf}}{\def\UrlFont{\small\ttfamily}}}
\makeatother
\urlstyle{leostyle}

%%%%%%%%%%%%%%%%
% Enumerations %
%%%%%%%%%%%%%%%%

\setlength{\pltopsep}{0.24em}
\setlength{\plpartopsep}{0em}
\setlength{\plitemsep}{0.24em}

% This should do what we want
%   \setdefaultenum{(i)}{(a)}{1.}{A}
% but it does not work for references, dropping the 
% parentheses.  The following hack does work.

\renewcommand{\theenumi}{(\roman{enumi})}
\renewcommand{\theenumii}{(\alph{enumii})}
\renewcommand{\theenumiii}{\arabic{enumiii}.}
\renewcommand{\theenumiv}{\Alph{enumiv}}

\renewcommand{\labelenumi}{\theenumi}
\renewcommand{\labelenumii}{\theenumii}
\renewcommand{\labelenumiii}{\theenumiii}
\renewcommand{\labelenumiv}{\theenumiv}

%%%%%%%%%%%%%%%%%%%%%%%%%
% Mathematical commands %
%%%%%%%%%%%%%%%%%%%%%%%%%

\renewcommand{\to}{\rightarrow}%         Right arrow
\newcommand{\into}{\hookrightarrow}%     Injection arrow
\newcommand{\onto}{\twoheadrightarrow}%  Surjection arrow

\providecommand{\abs}[1]{\lvert#1\rvert}%                  Absolute value
\providecommand{\absbig}[1]{\bigl\lvert#1\bigr\rvert}%     Absolute value
\providecommand{\absBig}[1]{\Bigl\lvert#1\Bigr\rvert}%     Absolute value
\providecommand{\absbigg}[1]{\biggl\lvert#1\biggr\rvert}%  Absolute value

\providecommand{\norm}[1]{\lVert#1\rVert}%               Norm
\providecommand{\normbig}[1]{\bigl\lVert#1\bigr\rVert}%  Norm
\providecommand{\normBig}[1]{\Bigl\lVert#1\Bigr\rVert}%  Norm

\providecommand{\floor}[1]{\left\lfloor#1\right\rfloor}%    Floor
\providecommand{\floorbig}[1]{\bigl\lfloor#1\bigr\rfloor}%  Floor
\providecommand{\floorBig}[1]{\Bigl\lfloor#1\Bigr\rfloor}%  Floor

\providecommand{\ceil}[1]{\left\lceil#1\right\rceil}%    Ceiling
\providecommand{\ceilbig}[1]{\bigl\lceil#1\bigr\rceil}%  Ceiling
\providecommand{\ceilBig}[1]{\Bigl\lceil#1\Bigr\rceil}%  Ceiling

\newcommand{\N}{\mathbf{N}}%  Natural numbers
\newcommand{\Z}{\mathbf{Z}}%  Integers
\newcommand{\Q}{\mathbf{Q}}%  Rationals

\allowdisplaybreaks[4]
%\numberwithin{equation}{section}

%%%%%%%%%%%%
% listings %
%%%%%%%%%%%%

\lstset{language=c}
\lstset{basicstyle=\ttfamily}
\lstset{keywordstyle=}
%\lstset{morekeywords={mpz_t,mpz_poly_t,fmpz_poly_t}}  TODO
\lstset{escapechar=\%}

%%%%%%%%%%%%%%%%%%%%%%%%%%%
% FLINT specific commands %
%%%%%%%%%%%%%%%%%%%%%%%%%%%

\newcommand{\code}{\lstinline}

%%%%%%%%%%%%%%%%%%%%%%%%%%%%%%%%%%%%%%%%%%%%%%%%%%%%%%%%%%%%%%%%%%%%%%%%%%%%%%%
% DOCUMENT                                                                    %
%%%%%%%%%%%%%%%%%%%%%%%%%%%%%%%%%%%%%%%%%%%%%%%%%%%%%%%%%%%%%%%%%%%%%%%%%%%%%%%

\begin{document}

%%%%%%%%%%%%%%%%%%%%%%%%%%%%%%%%%%%%%%%%%%%%%%%%%%%%%%%%%%%%%%%%%%%%%%%%%%%%%%%
% FRONTMATTER                                                                 %
%%%%%%%%%%%%%%%%%%%%%%%%%%%%%%%%%%%%%%%%%%%%%%%%%%%%%%%%%%%%%%%%%%%%%%%%%%%%%%%

\frontmatter

\maketitle

\tableofcontents

%%%%%%%%%%%%%%%%%%%%%%%%%%%%%%%%%%%%%%%%%%%%%%%%%%%%%%%%%%%%%%%%%%%%%%%%%%%%%%%
% MAINMATTER                                                                  %
%%%%%%%%%%%%%%%%%%%%%%%%%%%%%%%%%%%%%%%%%%%%%%%%%%%%%%%%%%%%%%%%%%%%%%%%%%%%%%%

\mainmatter

\chapter{Introduction}

FLINT is a C library of functions for doing number theory. It is highly 
optimised and can be compiled on numerous platforms.  FLINT also has the 
aim of providing support for multicore and multiprocessor computer 
architectures, though we do not yet provide this facility.

FLINT is currently maintained by William Hart of Warwick University in 
the UK.

As of version 1.1.0 FLINT supports 32 and 64 bit processors including 
x86, PPC, Alpha and Itanium processors, though in theory it compiles on any 
machine with GCC version 3.4 or later and with GMP version 4.2.1 or 
MPIR 0.9.0 or later.

FLINT is supplied as a set of modules, \code{fmpz}, \code{fmpz_poly}, etc., 
each of which can be linked to a C program making use of their functionality.

All of the functions in FLINT have a corresponding test function provided 
in an appropriately named test file.  For example, the function 
\code{fmpz_poly_add} located in \code{fmpz_poly/add.c} has test code in the 
file \code{fmpz_poly/test/t-add.c}.

\chapter{Installing FLINT}

\section{Building FLINT}

The easiest way to use FLINT is to build a shared library.  Simply download 
the FLINT tarball and untar it on your system.

FLINT requires GMP version 4.2.1 or later or MPIR version 0.9.0 or 
later (in GMP compatibility mode).  Set the environment variables 
\code{FLINT_GMP_LIB_DIR} and \code{FLINT_GMP_INCLUDE_DIR} to point 
to your GMP or MPIR library and include directories respectively. 
Alternatively you can set default values for these environment variables 
in the \code{flint_env} file.

Once the environment variables are set or defaults are set in 
\code{flint_env} simply type:

\begin{lstlisting}[language=bash]
source flint_env
\end{lstlisting}

in the main directory of the FLINT directory tree.  Finally type:

\begin{lstlisting}[language=bash]
make library
\end{lstlisting}

Move the library file \code{libflint.so}, \code{libflint.dll} or 
\code{libflint.dylib} (depending on your platform) into your library 
path and move all the \code{.h} files in the main directory of FLINT 
into your include path.

Now to use FLINT, simply include the appropriate header files for 
the FLINT modules you wish to use in your C program.  Then compile 
your program, linking against the FLINT library and GMP/MPIR with 
the options \code{-lflint -lgmp}.

\section{Test code}

Each module of FLINT has an extensive associated test module.  We 
strongly recommend running the test programs before relying on results 
from FLINT on your system. 

To make and run the test programs, simply type:

\begin{lstlisting}[language=bash]
make check
\end{lstlisting}

in the main FLINT directory.

\section{Reporting bugs}

The maintainer wishes to be made aware of any and all bugs.  Please send an 
email with your bug report to \url{hart_wb@yahoo.com}.

If possible please include details of your system, version of gcc, version 
of GMP/MPIR and precise details of how to replicate the bug.

Note that FLINT needs to be linked against version 4.2.1 or later of GMP 
or version 0.9.0 or later of MPIR (in GMP compatibility mode) and must be 
compiled with gcc version 3.4 or later.  In particular the compiler must be 
fully C99 compatible.

\section{Example programs}

FLINT comes with a number of example programs to demonstrate current and 
future FLINT features.  To build the example programs, type:

\begin{lstlisting}[language=bash]
make examples
\end{lstlisting}

The current example programs are:

\code{delta_qexp}  Computes the first $n$ terms of the delta function, e.g.\ 
\code{delta_qexp 1000000} will compute the first one million terms of the 
$q$-expansion of delta.

\code{BPTJCubes}  Implements the algorithm of Beck, Pine, Tarrant and Jensen 
for finding solutions to the equation $x^3+y^3+z^3 = k$.  This program 
outputs a file \code{output.log} containing parameters for reconstructing the 
first solution it finds, and then aborts.

\code{bernoulli_zmod} Compute many bernoulli numbers modulo a prime.  If no 
command line input is supplied it merely checks that the \code{bernoulli_zmod} 
function works for the first $2000$ primes.  If you specify an integer 
argument \code{n} on the command line, it computes the Bernoulli numbers 
$B_0, B_2, \dotsc, B_{p-1}$ modulo~$p$, where $p$ is the next prime from 
\code{n}.

\code{expmod}  Computes a very large modular exponentiation.  This is actually 
a basic pseudo primality test.

\code{Zmul}  Compares the output of the FLINT FFT with that of GMP for ever 
larger operands.

\code{thetaproduct}  Computes the congruent number theta function.  To run 
this you need to have openmp on your machine, you need a recent version of 
gcc (e.g. 4.3.x or 4.4.x) and you need to export \code{OMP_NUM_THREADS=16} 
or some factor of 16, depending on how many cores your machine has.  The 
code also expects a directory \code{storage} with \emph{plenty} of space 
where temporary files will be created.  Be warned that this code multiplies 
\emph{huge} integers which do not fit into memory and much disk space is 
used.  You also need a significant amount of memory on your machine, which 
must also be a 64 bit linux platform.  Parameters can be changed at the top 
of the file \code{thetaproduct.c}.  Primitive (squarefree) zeroes of the 
congruent number theta function curve will be computed up to 
\code{MOD * LIMIT} in the class $K$ modulo \code{MOD}.  At present 
\code{FILES1} and \code{FILES2} must be equal.  \code{LIMIT} must also be 
divisible by \code{BLOCK} and by \code{BUNDLE * FILES1}.  The code is not 
currently designed to correctly handle small problems. 

\section{FLINT macros}

In the file \code{flint.h} are various useful macros.

The macro constant \code{FLINT_BITS} is set at compile time to be the 
number of bits per limb on the machine.  FLINT requires it to be either 
32 or 64 bits.  Other architectures are not currently supported.

The macro constant \code{FLINT_D_BITS} is set at compile time to be the 
number of bits per double on the machine or the number of bits per limb, 
whichever is smaller.  This will have the value 53 or 32 on currently 
supported architectures.  Numerous functions using precomputed inverses 
only support operands up to \code{FLINT_D_BITS} bits, hence the macro.

\code{FLINT_ABS(x)} returns the absolute value of a \code{long x}.

\code{FLINT_MIN(x, y)} returns the minimum of two \code{long} or two 
\code{unsigned long} values \code{x} and \code{y}.

\code{FLINT_MAX(x, y)} returns the maximum of two \code{long} or two 
\code{unsigned long} values \code{x} and \code{y}.

\code{FLINT_BIT_COUNT(x)} returns the number of binary bits required 
to represent an \code{unsigned long x}.

\chapter{fmpz\_poly}

\input{fmpz_poly.tex}

%%%%%%%%%%%%%%%%%%%%%%%%%%%%%%%%%%%%%%%%%%%%%%%%%%%%%%%%%%%%%%%%%%%%%%%%%%%%%%%
% BACKMATTER                                                                  %
%%%%%%%%%%%%%%%%%%%%%%%%%%%%%%%%%%%%%%%%%%%%%%%%%%%%%%%%%%%%%%%%%%%%%%%%%%%%%%%

\backmatter

\end{document}
